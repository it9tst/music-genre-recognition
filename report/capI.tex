Un genere musicale è una categoria convenzionale che identifica e classifica i brani e le composizioni in base a criteri di affinità. Le musiche possono essere raggruppate in base alle loro convenzioni formali e stilistiche, alla tradizione in cui si inseriscono, allo spirito dei loro temi, alla loro destinazione o, se presente, al loro testo. L'indeterminatezza di alcuni di questi parametri rende spesso la divisione della musica in generi controversa e arbitraria. Un genere musicale può a sua volta dividersi in sottogeneri che ne ampliano la complessità. Fortunatamente, su Internet si possono trovare tanti \textit{dataset} adatti al progetto che si vuole svolgere. 

\section{Set dati FMA}
In un primo momento si è deciso di utilizzare il set dati \textit{FMA (Free Music Archive)} un set di dati open source e facilmente accessibile, adatto per valutare diverse attività di \textit{MIR (Music Information Retrieval)}. Esso contiene 8000 brani di 30 secondi l'uno divisi equamente in 8 generi:
\begin{itemize}
	\item Electronic
	\item Experimental
	\item Folk
	\item Hip-Hop
	\item Instrumental
	\item International
	\item Pop
	\item Rock
\end{itemize}
Inoltre viene dato in dotazione anche un file \textit{.csv} contenente tutti i metadata dei brani come ID, titolo, artista, genere, tags, etc.\\
Questo \textit{dataset} è stato utilizzato per i modelli basati su alberi decisionali, che si trovano nella cartella \textit{models}, che vanno dal numero 1 al numero 6.\\
Nota a margine, invece di 8000 brani se ne sono potuti utilizzare solo 7994 perchè sei file erano corrotti.

\section{Set dati GTZAN}
Successivamente si è utilizzato il set dati \textit{GTZAN} (detto anche il \textit{MNIST of sounds}) che, a differenza del primo, contiene 1000 brani di 30 secondi l'uno divisi in 100 file audio per ogni genere. Quindi questa volta abbiamo a disposizione 10 generi:
\begin{itemize}
	\item Blues
	\item Classical
	\item Country
	\item Disco
	\item Hip-Hop
	\item Jazz
	\item Metal
	\item Pop
	\item Reggae
	\item Rock
\end{itemize}
Il set di dati \textit{GTZAN} è il set di dati open source più utilizzato con il machine learning per il riconoscimento del genere musicale. I file sono stati raccolti tra il 2000 e il 2001 da una varietà di fonti tra cui CD personali, radio e registrazioni microfoniche, al fine di rappresentare una varietà di condizioni di registrazione.\\
Questo \textit{dataset} è stato utilizzato per i modelli basati su alberi decisionali, che si trovano nella cartella \textit{models}, che vanno dal numero 7 al numero 11.\\
Nota a margine, invece di 1000 brani se ne sono potuti utilizzare solo 999 perchè un file era corrotto.